
\documentclass[a4paper, 12pt]{article}

\usepackage{graphicx}
\usepackage{longtable}
\usepackage{rotating}
\usepackage{dcolumn}

\begin{document}
\title{ARE213 Problem Set \#1A}
\author{Peter Alstone \& Frank Proulx}
\maketitle

\section{Problem \#1}
\subsection{Part A}

Data records are excluded from the dataset based whether the following variables take the noted values \(as found in the data manual\):

\subsection{Part B}
We dropped all rows where any data were missing in that row.  One way that the data cleaning process could be improved would be to only remove records based on the variables of interest (as are determined in subsequent analysis) since missing values in fields that are not eventually used in the analysis do not pose a problem..  This would result in a more iterative approach, however, and increase workload on the researcher.

We used some exploratory analysis to understand if the records that were dropped due to missing data \textit{somewhere} in the record were representative.  First we compared a few simple summary statistics between the "full record" and "partial record" data on variables of interest for this analysis.  These are summarized in Table \ref{tab:compareMissingData}.  Better APGAR scores and lower incidence of smoking may be correlated with having full datasets, which indicates the people who have missing data may bias the sample.  We also used agnostic linear regression to understand the relationship between the presence of full records and three key variables: one-minute apgar (omaps), five-minute apgar (fmaps), and number of cigarettes smoked each day (cigar).  The results summarized in Table \ref{tab:lmMissingData} indicate there is statistical significance in each of the factors (i.e. all three are useful predictors for whether a person has a full data record) but also that the influence of the factors is small.  Figure \ref{fig:cigarFullData} shows the distribution in the number of cigarettes smoked by those with and without full records.  The distribution of values is basically the same (clusters around multiples of five up to 20, or, a "pack a day") between the two datasets.  

Overall, in spite of the bias from removing heavier smokers with lower apgar scores from the data, the overall number removed is relatively small and the size of the bias (indicated by the coefficients in the linear model) is relatively small.  

% Table created by stargazer v.4.0 by Marek Hlavac, Harvard University. E-mail: hlavac at fas.harvard.edu
% Date and time: Thu, Sep 19, 2013 - 16:27:57
\begin{table}[!htbp] \centering 
  \caption{Comparison of data with full records to those with missing data across key variables} 
  \label{tab:compareMissingData} 
  \footnotesize
\begin{tabular}{@{\extracolsep{5pt}} ccccccc} 
\\[-1.8ex]\hline 
\hline \\[-1.8ex] 
full.record & mean.omaps & sd.omaps & mean.fmaps & sd.fmaps & mean.cigar & sd.cigar \\ 
\hline \\[-1.8ex] 
FALSE & $7.905$ & $1.572$ & $8.880$ & $1.030$ & $3.945$ & $7.422$ \\ 
TRUE & $8.117$ & $1.260$ & $9.009$ & $0.707$ & $1.907$ & $5.297$ \\ 
\hline \\[-1.8ex] 
\end{tabular} 
\end{table} 


% Table created by stargazer v.4.0 by Marek Hlavac, Harvard University. E-mail: hlavac at fas.harvard.edu
% Date and time: Thu, Sep 19, 2013 - 16:32:14
\begin{table}[!htbp] \centering 
  \caption{Linear model results for predicting whether full records are present based on selected variable of interest in the dataset} 
  \label{tab:lmMissingData} 
\begin{tabular}{@{\extracolsep{5pt}}lc} 
\\[-1.8ex]\hline 
\hline \\[-1.8ex] 
 & \multicolumn{1}{c}{\textit{Dependent variable:}} \\ 
\cline{2-2} 
\\[-1.8ex] & full.record \\ 
\hline \\[-1.8ex] 
 omaps & 0.002$^{***}$ \\ 
  & (0.001) \\ 
  & \\ 
 fmaps & 0.007$^{***}$ \\ 
  & (0.001) \\ 
  & \\ 
 cigar & $-$0.003$^{***}$ \\ 
  & (0.0001) \\ 
  & \\ 
 Constant & 0.882$^{***}$ \\ 
  & (0.007) \\ 
  & \\ 
\hline \\[-1.8ex] 
Observations & 119,384 \\ 
R$^{2}$ & 0.007 \\ 
Adjusted R$^{2}$ & 0.007 \\ 
Residual Std. Error & 0.195 \\ 
F Statistic & 276.305 \\ 
\hline 
\hline \\[-1.8ex] 
\textit{Note:}  & \multicolumn{1}{r}{$^{*}$p$<$0.1; $^{**}$p$<$0.05; $^{***}$p$<$0.01} \\ 
\normalsize 
\end{tabular} 
\end{table}


\begin{figure}[!h] %  figure placement: here, top, bottom, or page
   \centering
   \includegraphics[width=4in]{img/cigar-by-record-type.pdf} 
   \caption{Cigarette use rate by presence of full data record.}
   \label{fig:cigarFullData}
\end{figure}

\subsection{Part C}
The summary table for the remaining data after cleaning is below.
% Code for the summary table from file summarytable.tex starts here...
% latex.default(summarytable) 
%

% latex.default(title = "variable", file = "clean-summary.tex",      cbind(var.labels, summarytable), caption = "Summary of clean Data",      vbar = TRUE, size = "footnotesize") 
%
\begin{table}[!tbp]
\footnotesize
\caption{Summary of clean data}
\label{tab:summaryClean} 
\begin{center}
\begin{tabular}{|l|l|r|r|r|r|r|}
\hline\hline
\multicolumn{1}{|l|}{variable}&\multicolumn{1}{c|}{var.labels}&\multicolumn{1}{c|}{var}&\multicolumn{1}{c|}{n}&\multicolumn{1}{c|}{mean}&\multicolumn{1}{c|}{sd}&\multicolumn{1}{c|}{se}\tabularnewline
\hline
rectype&record type&$ 1$&$114610$&$   1.26$&$  0.44$&$0.00$\tabularnewline
pldel3&facility of birth recode&$ 2$&$114610$&$   1.02$&$  0.13$&$0.00$\tabularnewline
birattnd&attendant at birth&$ 3$&$114610$&$   1.20$&$  0.56$&$0.00$\tabularnewline
cntocpop&county of occurence&$ 4$&$114610$&$   1.44$&$  1.14$&$0.00$\tabularnewline
stresfip&state of residence&$ 5$&$114610$&$  41.74$&$  2.17$&$0.01$\tabularnewline
dmage&age of mother&$ 6$&$114610$&$  27.76$&$  5.70$&$0.02$\tabularnewline
ormoth&hispanic origin of mother&$ 7$&$114610$&$   0.09$&$  0.52$&$0.00$\tabularnewline
mrace3&race of mother recode&$ 8$&$114610$&$   1.26$&$  0.66$&$0.00$\tabularnewline
dmeduc&detailed educ of mother&$ 9$&$114610$&$  13.21$&$  2.27$&$0.01$\tabularnewline
dmar&marital status of mother&$10$&$114610$&$   1.25$&$  0.43$&$0.00$\tabularnewline
adequacy&adequacy of care recode&$11$&$114610$&$   1.30$&$  0.55$&$0.00$\tabularnewline
nlbnl&number of live births, now living&$12$&$114610$&$   0.97$&$  1.15$&$0.00$\tabularnewline
dlivord&number of live births, now dead&$13$&$114610$&$   1.99$&$  1.17$&$0.00$\tabularnewline
dtotord&detail total birth order&$14$&$114610$&$   2.42$&$  1.52$&$0.00$\tabularnewline
totord9&total birth order recode&$15$&$114610$&$   2.41$&$  1.46$&$0.00$\tabularnewline
monpre&month pregnancy prenatal care began&$16$&$114610$&$   2.50$&$  1.33$&$0.00$\tabularnewline
nprevist&total number of prenatal visits&$17$&$114610$&$  11.15$&$  3.52$&$0.01$\tabularnewline
disllb&interval since last live birth&$18$&$114610$&$ 350.41$&$362.33$&$1.07$\tabularnewline
isllb10&interval since last live birth recode&$19$&$114610$&$   3.32$&$  3.19$&$0.01$\tabularnewline
dfage&age of father&$20$&$114610$&$  30.06$&$  6.41$&$0.02$\tabularnewline
orfath&hispanic origin of father&$21$&$114610$&$   0.09$&$  0.53$&$0.00$\tabularnewline
dfeduc&educ of father detail&$22$&$114610$&$  13.28$&$  2.33$&$0.01$\tabularnewline
birmon&month of birth&$23$&$114610$&$   6.47$&$  3.39$&$0.01$\tabularnewline
weekday&day of week child born&$24$&$114610$&$   4.05$&$  1.88$&$0.01$\tabularnewline
dgestat&gestation -- detail in weeks&$25$&$114610$&$  39.15$&$  2.44$&$0.01$\tabularnewline
csex&sex of child&$26$&$114610$&$   1.49$&$  0.50$&$0.00$\tabularnewline
dbrwt&birthweight in grams&$27$&$114610$&$3373.29$&$585.17$&$1.73$\tabularnewline
dplural&plurality&$28$&$114610$&$   1.03$&$  0.17$&$0.00$\tabularnewline
omaps&one minute agpar score&$29$&$114610$&$   8.12$&$  1.26$&$0.00$\tabularnewline
fmaps&five minute agpar score&$30$&$114610$&$   9.01$&$  0.71$&$0.00$\tabularnewline
clingest&clinical estimate of gestation&$31$&$114610$&$  39.11$&$  2.06$&$0.01$\tabularnewline
delmeth5&method of delivery&$32$&$114610$&$   1.55$&$  1.01$&$0.00$\tabularnewline
anemia&anemia mother&$33$&$114610$&$   1.99$&$  0.10$&$0.00$\tabularnewline
cardiac&cardiac disease mother&$34$&$114610$&$   1.99$&$  0.08$&$0.00$\tabularnewline
lung&acute or chronic lung disease mother&$35$&$114610$&$   1.99$&$  0.08$&$0.00$\tabularnewline
diabetes&diabetes mother&$36$&$114610$&$   1.97$&$  0.16$&$0.00$\tabularnewline
herpes&genital herpes mother&$37$&$114610$&$   1.99$&$  0.08$&$0.00$\tabularnewline
chyper&chronic hypertension&$38$&$114610$&$   1.99$&$  0.09$&$0.00$\tabularnewline
phyper&pregnancy related hypertension&$39$&$114610$&$   1.97$&$  0.17$&$0.00$\tabularnewline
pre4000&previous infant 4000 or more grams&$40$&$114610$&$   1.99$&$  0.12$&$0.00$\tabularnewline
preterm&previous preterm infant&$41$&$114610$&$   1.99$&$  0.12$&$0.00$\tabularnewline
tobacco&tobacco use during pregnancy&$42$&$114610$&$   1.84$&$  0.37$&$0.00$\tabularnewline
cigar&average number of cigarettes per day&$43$&$114610$&$   1.91$&$  5.30$&$0.02$\tabularnewline
cigar6&average number of cigarettes per day recode&$44$&$114610$&$   0.35$&$  0.86$&$0.00$\tabularnewline
alcohol&alcohol use during pregnancy&$45$&$114610$&$   1.99$&$  0.10$&$0.00$\tabularnewline
drink&average number of drinks per week&$46$&$114610$&$   0.03$&$  0.62$&$0.00$\tabularnewline
drink5&average number of drinks recode&$47$&$114610$&$   0.02$&$  0.23$&$0.00$\tabularnewline
wgain&weight gain&$48$&$114610$&$  30.36$&$ 11.88$&$0.04$\tabularnewline
full.record*&full record present&$49$&$114610$&$   1.00$&$  0.00$&$0.00$\tabularnewline
\hline
\end{tabular}
\end{center}
\end{table}

 % need to fix this table so it is more visible...currently runs off the page.  

\section{Problem \#2}

\subsection{Part A}

The table below shows the mean differences between smoking and non-smoking mothers for one-minute APGAR scores (ompas), five-minute (fmaps), and birth weight in grams (dbrwt).  Unconditioned on the other variables, there is no statistically significant difference in APGAR score but a significant difference is present in birth weight\footnote{Welch Two Sample t-test, alternative hypothesis: true difference in means is not equal to 0; p-value less than 2.2e-16, 95 percent confidence interval: -249.5463 to -231.4093}.   

% Table created by stargazer v.4.0 by Marek Hlavac, Harvard University. E-mail: hlavac at fas.harvard.edu
% Date and time: Sat, Sep 21, 2013 - 00:11:19
\begin{table}[!htbp] \centering 
  \caption{Comparison of key birthing infant health indicators for different maternal smoking status} 
  \label{} 
\begin{tabular}{@{\extracolsep{5pt}} cccc} 
\\[-1.8ex]\hline 
\hline \\[-1.8ex] 
tobacco & mean.omaps & mean.fmaps & mean.dbrwt \\ 
\hline \\[-1.8ex] 
smoker & 8.10 & 9.01 & 3171 \\ 
nonsmoker & 8.12 & 9.01 & 3412 \\ 
difference & 0.017 & 0.0001 & 240.5 \\ 
\hline \\[-1.8ex] 
\normalsize 
\end{tabular} 
\end{table} 


\subsection{Part B}

The average treatment effect (ATE) of maternal smoking can only be determined by comparing the unadjusted difference in mean birth weight of infants \textbf{if their mothers were randomly assigned into treatment (a smoking habit during pregnancy) or the assignment / selection to treatment is as good as random}.  This is obviously not possible or even palatable for a variety of practical and ethical reasons to verify with RCT so an alternative approach to identifying the ATE that controls for observables is the next-best option.  If we assume that smoking habits are randomly assigned among pregnant mothers, it can be "safe" to use the unadjusted difference in weight as a predictor of ATE without conditioning on observables as long as there are not any significant differences in the smoking and non-smoking groups that also influence birth weight.  In the next set of steps we explore other factors that may influence birth weight and if they are also related to smoking status.  
  

\paragraph{ATE using unadjusted differences:}
If we were to assume that smoking is in fact randomly assigned, the mean difference in birth weight caused by smoking between infants whose mothers smoke and those who do not is 240 grams (with a 95\% confidence interval of 230 - 250 grams). Infants whose mother smoked have about  7\% lower birth weight than those who did not. \newline{}


\paragraph{Identifying potential confounding factors:}
We used deductive logic and graphical exploration to understand factors that may influence birth weight and should be controlled for if the tobacco users / non-users have distributions that are not identical (or very similar) between them.  Several (but not all) of the factors that we identified as potential candidates are summarized in the Table \ref{tab:xtabsTobacco}.  We omitted many that did not show a relationship between the factor and birth weight for brevity.  The results show that most of the potential factors related to birth weight do not appear likely to be also related to smoking status.  \newline

% latex.default(cstats, title = title, caption = caption, rowlabel = rowlabel,      col.just = col.just, numeric.dollar = FALSE, insert.bottom = legend,      rowname = lab, dcolumn = dcolumn, extracolheads = extracolheads,      extracolsize = Nsize, ...) 
%
\begin{table}[!tbp]
\caption{Contingency table for a range of factors by tobacco use status\label{tab:xtabTobacco}} 
\begin{center}
\begin{tabular}{lccc}
\hline\hline
\multicolumn{1}{l}{}&\multicolumn{1}{c}{smoker}&\multicolumn{1}{c}{nonsmoker}&\multicolumn{1}{c}{Combined}\tabularnewline
&\multicolumn{1}{c}{{\scriptsize $N=18266$}}&\multicolumn{1}{c}{{\scriptsize $N=96344$}}&\multicolumn{1}{c}{{\scriptsize $N=114610$}}\tabularnewline
\hline
race~of~mother~recode~:~White&87\%~{\scriptsize~(15876)}&86\%~{\scriptsize~(82748)}&86\%~{\scriptsize~(98624)}\tabularnewline
~~~~Other&~0\%~{\scriptsize~(~~~69)}&~2\%~{\scriptsize~(~2202)}&~2\%~{\scriptsize~(~2271)}\tabularnewline
~~~~Black&13\%~{\scriptsize~(~2321)}&12\%~{\scriptsize~(11394)}&12\%~{\scriptsize~(13715)}\tabularnewline
sex~of~child~:~Male&52\%~{\scriptsize~(~9462)}&51\%~{\scriptsize~(49505)}&51\%~{\scriptsize~(58967)}\tabularnewline
~~~~Female&48\%~{\scriptsize~(~8804)}&49\%~{\scriptsize~(46839)}&49\%~{\scriptsize~(55643)}\tabularnewline
marital~status~of~mother~:~Married&52\%~{\scriptsize~(~9459)}&79\%~{\scriptsize~(76368)}&75\%~{\scriptsize~(85827)}\tabularnewline
~~~~Unmarried&48\%~{\scriptsize~(~8807)}&21\%~{\scriptsize~(19976)}&25\%~{\scriptsize~(28783)}\tabularnewline
plurality~:~Singleton&98\%~{\scriptsize~(~17860)}&97\%~{\scriptsize~(~93694)}&97\%~{\scriptsize~(111554)}\tabularnewline
~~~~Twin&~2\%~{\scriptsize~(~~~400)}&~3\%~{\scriptsize~(~~2503)}&~3\%~{\scriptsize~(~~2903)}\tabularnewline
~~~~Triplet&~0\%~{\scriptsize~(~~~~~6)}&~0\%~{\scriptsize~(~~~135)}&~0\%~{\scriptsize~(~~~141)}\tabularnewline
~~~~Quadruplet&~0\%~{\scriptsize~(~~~~~0)}&~0\%~{\scriptsize~(~~~~12)}&~0\%~{\scriptsize~(~~~~12)}\tabularnewline
alcohol~use~during~pregnancy~:~Drinker&~~3\%~{\scriptsize~(~~~639)}&~~0\%~{\scriptsize~(~~~472)}&~~1\%~{\scriptsize~(~~1111)}\tabularnewline
~~~~Nondrinker&~97\%~{\scriptsize~(~17627)}&100\%~{\scriptsize~(~95872)}&~99\%~{\scriptsize~(113499)}\tabularnewline
pregnancy~related~hypertension~:~1&~2\%~{\scriptsize~(~~~369)}&~3\%~{\scriptsize~(~~3149)}&~3\%~{\scriptsize~(~~3518)}\tabularnewline
~~~~2&98\%~{\scriptsize~(~17897)}&97\%~{\scriptsize~(~93195)}&97\%~{\scriptsize~(111092)}\tabularnewline
chronic~hypertension~:~1&~1\%~{\scriptsize~(~~~120)}&~1\%~{\scriptsize~(~~~764)}&~1\%~{\scriptsize~(~~~884)}\tabularnewline
~~~~2&99\%~{\scriptsize~(~18146)}&99\%~{\scriptsize~(~95580)}&99\%~{\scriptsize~(113726)}\tabularnewline
cardiac~disease~mother~:~1&~1\%~{\scriptsize~(~~~111)}&~1\%~{\scriptsize~(~~~677)}&~1\%~{\scriptsize~(~~~788)}\tabularnewline
~~~~2&99\%~{\scriptsize~(~18155)}&99\%~{\scriptsize~(~95667)}&99\%~{\scriptsize~(113822)}\tabularnewline
diabetes~mother~:~1&~3\%~{\scriptsize~(~~~490)}&~3\%~{\scriptsize~(~~2587)}&~3\%~{\scriptsize~(~~3077)}\tabularnewline
~~~~2&97\%~{\scriptsize~(~17776)}&97\%~{\scriptsize~(~93757)}&97\%~{\scriptsize~(111533)}\tabularnewline
previous~infant~4000~or~more~grams~:~1&~1\%~{\scriptsize~(~~~154)}&~2\%~{\scriptsize~(~~1506)}&~1\%~{\scriptsize~(~~1660)}\tabularnewline
~~~~2&99\%~{\scriptsize~(~18112)}&98\%~{\scriptsize~(~94838)}&99\%~{\scriptsize~(112950)}\tabularnewline
month~pregnancy~prenatal~care~began&{\scriptsize 2~}{2 }{\scriptsize 3} &{\scriptsize 2~}{2 }{\scriptsize 3} &{\scriptsize 2~}{2 }{\scriptsize 3} \tabularnewline
age~of~mother&{\scriptsize 22~}{26 }{\scriptsize 30} &{\scriptsize 24~}{28 }{\scriptsize 32} &{\scriptsize 24~}{28 }{\scriptsize 32} \tabularnewline
clinical~estimate~of~gestation&{\scriptsize 38~}{40 }{\scriptsize 40} &{\scriptsize 38~}{40 }{\scriptsize 40} &{\scriptsize 38~}{40 }{\scriptsize 40} \tabularnewline
weight~gain&{\scriptsize 20~}{29 }{\scriptsize 37} &{\scriptsize 24~}{30 }{\scriptsize 37} &{\scriptsize 23~}{30 }{\scriptsize 37} \tabularnewline
\hline
\end{tabular}
\end{center}
\noindent {\scriptsize $a$\ }{$b$\ }{\scriptsize $c$\ } represent the lower quartile $a$, the median $b$, and the upper quartile $c$\ for continuous variables.\\Numbers after percents are frequencies.\end{table}

 %Table with sample cross tabs for smoking status and a variety of factors.



The factors we identify as having an impact on birth weight AND being related to smoking status are:

\begin{itemize}

\item \textbf{Maternal Age} is different between the smoking / non-smoking group and is related to birth weight.  The median pregnant smoker is two years younger than the median non-smoker.  There is also a relationship between age and birth weight (where older mothers up to age 31-32 or so tend to have heavier babies).  The relationship between maternal age and birth weight along with the distributions in age for smokers and non-smokers is shown in Figure \ref{fig:bwAge}

\item \textbf{Marital Status} is also different between the smoking and non-smoking groups: single mothers are more likely to smoke in pregnancy.  In the whole sample the fraction of women who are married is 75\% but in the ``smoker'' subsample it is only 52\%.  There is also a relationship between marital status and birth weight whereby married mothers tend to have slightly heavier babies.  These relationships are shown in Figure \ref{fig:bwMar}.  If one believes that being married leads to less stress for mothers and/or better resources and support it is possible that marital status is a proxy for other determinants of infant weight.  However, as is also shown in the Figure (bottom panel) there are different distributions in maternal age between married and unmarried women, with a relationship that suggests age may be a stronger determining factor since single mothers are typically younger than married mothers.   

\item \textbf{Maternal weight gain (less certain)} is related to infant weight at birth but as we note is not as certain in terms of being related strongly with smoking status.  The median weight gain is quite similar between the two smoking status groups (29 lbs for smokers vs. 30 lbs for non-smokers) but there is a larger difference in the 25th percentile weight (20 vs. 24 lbs.).  The relationship between maternal weight gain and infant weight gain along with the distribution in maternal gain by smoking status is summarized in Figure \ref{fig:bwGain}

\begin{figure}[htbp]
\begin{center}
\includegraphics{img/bw-age.pdf}
\caption{(Top) The relationship between Maternal Age and Birth Weight with a GAM fit to the data and 95\% confidence interval estimate in grey.  Actual data are omitted to show the average trend more clearly.  (Bottom panels) A comparison in the distribution of Maternal Age for smokers and non-smokers shows how smokers tend to be younger mothers.}
\label{fig:bwAge}
\end{center}
\end{figure}

\begin{figure}[htbp]
\begin{center}
\includegraphics{img/bw-mar.pdf}
\caption{(Top) Boxplots for birth weight by marital status of the mother. (Bottom) Distribution in maternal age between married and unmarried women.}
\label{fig:bwMar}
\end{center}
\end{figure}

\begin{figure}[htbp]
\begin{center}
\includegraphics{img/bw-gain.pdf}
\caption{(Top) The relationship between Maternal Weight Gain and Birth Weight with a GAM fit to the data and 95\% confidence interval estimate in grey.  Actual data are omitted to show the average trend more clearly.  (Bottom panels) A comparison in the distribution of Maternal Weight Gain for smokers and non-smokers shows how smokers tend to have (slightly) less weight gain.}
\label{fig:bwGain}
\end{center}
\end{figure}

\end{itemize}
Because of the factors we identified the assumption that smoking is randomly assigned in the pop ulation(and using unadjusted mean differences) is not tenable for obtaining an accurate prediction of ATE. 


\subsection{Part C}
The position that smoking status is randomly assigned may have some rational basis, but we cannot claim complete randomness.  Consider the following:

\begin{itemize}
\item This study was conducted in 1993, decades after the link between smoking and poor infant health was established and widely publicized in both the scientific literature and (more importantly) the popular media.  While there is a link between maternal educational attainment (smokers tend to have less education, slightly), this can largely be explained by the age of the mothers (many of whom are simply too young to have graduated college, etc.).  This education gap could potentially explain a difference in awareness but we posit it is probably a poor proxy.  It is reasonable to expect that the vast majority of mothers in the sample know about the link between smoking during pregnancy and poor infant health outcomes, and that the smoking and non-smoking mothers both have the same maternal drive to protect their unborn infants.

\item Furthermore, even if the popular exposure were different between smokers and non-smokers, it is standard practice during neonatal care to receive messages about the value of not smoking.  Both smokers and non-smokers presumably received roughly the same level of neonatal care (as measured by the month at which care began).  

\item If one accepts that awareness about smoking risk and the level of maternal protection drive is the same in both groups, perhaps the only remaining factors are those that underly addiction: genetic predisposition and environmental factors.  It is possible, but by no means certain, that the genetic factors (at least) are essentially randomly distributed between people.  However, the underlying environmental factors that lead to addiction are not likely randomly distributed and may be correlated with both smoking and other maternal behavior that could lead to lower birth weight.  

\item Based on the above we posit that there is an element of randomness (genetic factors) associated with smoking but that environmental factors (education, upbringing, social pressures) also contribute strongly to both smoking status and other factors that could cause low birth weight (like pregnancies earlier in life).  

\end{itemize}

Based on our analysis and reasoning, there are two key factors that are predetermined, contribute to birth weight, and also are biased by the "smoking" treatment: Maternal Age and Marital Status.  Other factors that are predetermined but do not meet the other two criteria (contributing to weight AND biased by smoking treatment comported to the larger sample) include the sex of the baby, level of prenatal care, age of the father (to the extent that there are genetic causes), gestational time, state of residence, infant plurality, etc.  There are also factors that are not predetermined and seem to be (potentially) closely linked with treatment status, such as alcohol use in utero (which was a small fraction of the whole population), maternal weight gain, pregnancy related hypertension, and incidences of lung disease.  \newline

In general the predetermined factors are those that can be completely extricated from the treatment, i.e., those that could be changed for a particular individual without changing the treatment category.  Factors that are not predetermined, and therefore cannot be pulled apart from treatment effects, are those that derive at least partly from the smoking status of the mother or underlying factors of smoking status.  The ultimate goal is to identify predetermined factors so they can be controlled in the regression to capture only factors that can be effected by the treatment.  \newline

The policy relevance of the analysis is identifying what the potential positive outcomes can be from targeted smoking cessation programs on pregnant (or soon to be pregnant) mothers, so one must assume that we cannot change other parts of her life leading up to the pregnancy.  

\subsection{Part D}

Selection on observables strategies for teasing out causality guide us to identify all the observable factors (except for the treatment--smoking in this case) that could lead to the outcome (birth weight) and to "correct" for these using statistical techniques before the test for smoking.  We identified three additional key factors above that we will use in a set of simple linear models to understand how different factors influence birth weight: Maternal Age, Marital Status, and Maternal Weight Gain.  We include maternal weight gain with prejudice on how to interpret it because it too could be linked with smoking status (anecdotal evidence indicates that smoking tends to suppress weight gain and that smoking cessation could lead to temporary weight gain).  Table \ref{bwLMNoTob} and \ref{bwLMWithTob} below summarize the suite of linear models we used to explore the relationships between various potential causal factors and birth weight\newline

The outcome of the models indicates that each of the three additional factors (in addition to tobacco use) is statistically significant in terms of predicting birth weight.  However, because maternal weight gain is not a clearly predetermined factor we choose to ignore models that include it (although it is interesting to examine them).  The "best" model that only includes Maternal Age and Marital Status as conditioning variables along with tobacco use.  \newline

By introducing conditioning variables in a linear regression we downgrade the estimate for average treatment effect from 240 grams to about 200 grams because smokers tended to be younger and unmarried (both significantly decrease birth weight).  



% TABLE Birthweight without tobacco.
% Table created by stargazer v.4.0 by Marek Hlavac, Harvard University. E-mail: hlavac at fas.harvard.edu
% Date and time: Mon, Sep 23, 2013 - 11:24:05
% Requires LaTeX packages: dcolumn 
\begin{sidewaystable}[!htbp] \centering 
  \caption{Birth weight linear models without including Tobacco factors} 
  \label{tab:bwLMNoTab} 
\footnotesize 
\begin{tabular}{@{\extracolsep{5pt}}lD{.}{.}{-3} D{.}{.}{-3} D{.}{.}{-3} } 
\\[-1.8ex]\hline 
\hline \\[-1.8ex] 
\\[-1.8ex] & \multicolumn{3}{c}{Birth Weight} \\ 
\\[-1.8ex] & \multicolumn{1}{c}{(1)} & \multicolumn{1}{c}{(2)} & \multicolumn{1}{c}{(3)}\\ 
\hline \\[-1.8ex] 
 Maternal Age & 9.536^{***} & 2.888^{***} & 4.246^{***} \\ 
  & (0.302) & (0.341) & (0.334) \\ 
  & & & \\ 
 Marital Status (unmarried) &  & -183.626^{***} & -175.893^{***} \\ 
  &  & (4.479) & (4.384) \\ 
  & & & \\ 
 Weight Gain &  &  & 10.042^{***} \\ 
  &  &  & (0.141) \\ 
  & & & \\ 
 Constant & 3,108.613^{***} & 3,339.251^{***} & 2,994.784^{***} \\ 
  & (8.558) & (10.190) & (11.081) \\ 
  & & & \\ 
\textit{N} & \multicolumn{1}{c}{114,610} & \multicolumn{1}{c}{114,610} & \multicolumn{1}{c}{114,610} \\ 
R$^{2}$ & \multicolumn{1}{c}{0.009} & \multicolumn{1}{c}{0.023} & \multicolumn{1}{c}{0.064} \\ 
Adjusted R$^{2}$ & \multicolumn{1}{c}{0.009} & \multicolumn{1}{c}{0.023} & \multicolumn{1}{c}{0.064} \\ 
Residual Std. Error & \multicolumn{1}{c}{582.649 (df = 114608)} & \multicolumn{1}{c}{578.426 (df = 114607)} & \multicolumn{1}{c}{566.024 (df = 114606)} \\ 
F Statistic & \multicolumn{1}{c}{996.918$^{***}$ (df = 1; 114608)} & \multicolumn{1}{c}{1,346.077$^{***}$ (df = 2; 114607)} & \multicolumn{1}{c}{2,629.910$^{***}$ (df = 3; 114606)} \\ 
\hline 
\hline \\[-1.8ex] 
\textit{Notes:} & \multicolumn{3}{r}{$^{***}$Significant at the 1 percent level.} \\ 
 & \multicolumn{3}{r}{$^{**}$Significant at the 5 percent level.} \\ 
 & \multicolumn{3}{r}{$^{*}$Significant at the 10 percent level.} \\ 
\normalsize 
\end{tabular} 
\end{sidewaystable} 



%TABLE Birthweight WITH tobacco
% Table created by stargazer v.4.0 by Marek Hlavac, Harvard University. E-mail: hlavac at fas.harvard.edu
% Date and time: Mon, Sep 23, 2013 - 11:35:16
% Requires LaTeX packages: dcolumn 
\begin{sidewaystable}[!htbp] \centering 
  \caption{Birth weight linear models including Tobacco factors} 
  \label{tab:bwLMWithTob} 
\footnotesize 
\begin{tabular}{@{\extracolsep{5pt}}lD{.}{.}{-3} D{.}{.}{-3} D{.}{.}{-3} } 
\\[-1.8ex]\hline 
\hline \\[-1.8ex] 
\\[-1.8ex] & \multicolumn{3}{c}{Birth Weight} \\ 
\\[-1.8ex] & \multicolumn{1}{c}{(1)} & \multicolumn{1}{c}{(2)} & \multicolumn{1}{c}{(3)}\\ 
\hline \\[-1.8ex] 
 Maternal Age & 4.058^{***} & 2.716^{***} & 7.781^{***} \\ 
  & (0.332) & (0.338) & (0.301) \\ 
  & & & \\ 
 Marital Status (unmarried) & -141.095^{***} & -146.507^{***} &  \\ 
  & (4.444) & (4.538) &  \\ 
  & & & \\ 
 Weight Gain & 9.850^{***} &  &  \\ 
  & (0.140) &  &  \\ 
  & & & \\ 
 Tobacco (nonsmoker) & 183.678^{***} & 195.101^{***} & 225.823^{***} \\ 
  & (4.668) & (4.764) & (4.690) \\ 
  & & & \\ 
 Constant & 2,842.679^{***} & 3,170.698^{***} & 2,967.483^{***} \\ 
  & (11.666) & (10.921) & (8.965) \\ 
  & & & \\ 
\textit{N} & \multicolumn{1}{c}{114,610} & \multicolumn{1}{c}{114,610} & \multicolumn{1}{c}{114,610} \\ 
R$^{2}$ & \multicolumn{1}{c}{0.077} & \multicolumn{1}{c}{0.037} & \multicolumn{1}{c}{0.028} \\ 
Adjusted R$^{2}$ & \multicolumn{1}{c}{0.077} & \multicolumn{1}{c}{0.037} & \multicolumn{1}{c}{0.028} \\ 
Residual Std. Error & \multicolumn{1}{c}{562.240 (df = 114605)} & \multicolumn{1}{c}{574.242 (df = 114606)} & \multicolumn{1}{c}{576.845 (df = 114607)} \\ 
F Statistic & \multicolumn{1}{c}{2,386.185$^{***}$ (df = 4; 114605)} & \multicolumn{1}{c}{1,469.452$^{***}$ (df = 3; 114606)} & \multicolumn{1}{c}{1,667.935$^{***}$ (df = 2; 114607)} \\ 
\hline 
\hline \\[-1.8ex] 
\textit{Notes:} & \multicolumn{3}{r}{$^{***}$Significant at the 1 percent level.} \\ 
 & \multicolumn{3}{r}{$^{**}$Significant at the 5 percent level.} \\ 
 & \multicolumn{3}{r}{$^{*}$Significant at the 10 percent level.} \\ 
\normalsize 
\end{tabular} 
\end{sidewaystable}

\pagebreak
\section{Appendix}
\begin{verbatim}
### This is Peter Alstone & Frank Proulx's solution to ARE213 PS1a
## Data is in the file "ps1.dta"


## working directories --------

# Peter
# setwd("~/Google Drive/ERG/Classes/ARE213/are213/ps1")

# Frank
setwd("~/media/frank/Data1/documents/School/Berkeley/Fall13/ARE213/are213/ps1")

## PACKAGES -------------

library(foreign) #this is to read in Stata data
library(Hmisc)
library(psych)
library(stargazer)
library(ggplot2) # for neato plotting tools
library(plyr) # for nice data tools like ddply
library(car) # "companion for applied regression" - recode fxn, etc.
library(gmodels) #for Crosstabs

# custom functions
source("../util/are213-func.R")
source("../util/watercolor.R") # for watercolor plots


## DATA -------------

ps1.data <- read.dta(file="ps1.dta")
#changed name of object from "data" to avoid ambiguity issues since "data" is often embedded in functions as a general object

print(nrow(ps1.data))



## Problem 1a: Fix missing values --------
## The following are the error codes for each of the 15 variables that need fixing:
# For cardiac - alcohol: "8" means missing record
# cardiac: 9
# lung: 9
# diabetes: 9
# herpes: 9
# chyper: 9
# phyper: 9
# pre4000: 9
# preterm: 9
# tobacco: 9
# cigar: 99
# cigar6: 6
# alcohol: 9
# drink: 99
# drink5: 5
# wgain: 99

# Identify which records have full data, then add a column to indicate full records or not
full.record.flag <- which(ps1.data$cardiac != 9 &
                            ps1.data$cardiac != 8 &
                            ps1.data$lung != 9 &
                            ps1.data$lung != 8 &
                            ps1.data$diabetes !=9 &
                            ps1.data$diabetes !=8 &
                            ps1.data$herpes != 9 &
                            ps1.data$herpes != 8 &
                            ps1.data$chyper != 9 &
                            ps1.data$chyper != 8 &
                            ps1.data$phyper != 9 &
                            ps1.data$phyper != 8 &
                            ps1.data$pre4000 !=9 &
                            ps1.data$pre4000 !=8 &
                            ps1.data$preterm != 9 &
                            ps1.data$preterm != 8 &
                            ps1.data$tobacco != 9 &
                            ps1.data$cigar != 99 &
                            ps1.data$cigar6 !=6 &
                            ps1.data$alcohol != 9 &
                            ps1.data$drink != 99 &
                            ps1.data$drink5 !=5 &
                            ps1.data$wgain !=99
                          )

# Column with flags for full records
ps1.data$full.record <- FALSE # initialize column as F
ps1.data$full.record[full.record.flag] <- TRUE #reassign level to T for full records


# Problem 1b: Describe dropped levels --------

# replace error rows in cigar with NA so they don't interfere with other calcs on influence of dropped values.
error.cigar <- which(ps1.data$cigar == 99)
ps1.data$cigar[error.cigar] <- NA

# compare records on things that (might) matter for this analysis...apgar, smoking, etc.
ps1.compare.records <- ddply(ps1.data, .(full.record), summarize,
                             mean.omaps = mean(omaps),
                             sd.omaps = sd(omaps), 
                             mean.fmaps = mean(fmaps),
                             sd.fmaps = sd(fmaps),
                             mean.cigar = mean(cigar, na.rm = TRUE), 
                             sd.cigar = sd(cigar, na.rm = TRUE)
                             )

#--> RESULT: There appears to be a variation in the mean cigarette use between groups, but with large standard deviation.

# Print result table for comparison
stargazer(ps1.compare.records, summary=FALSE)

# Plot to explore if missing value people smoke more cigarettes
pdf(file="img/cigar-by-record-type.pdf", width = 7, height = 6)
ggplot(ps1.data, aes(cigar)) + geom_density() + facet_grid(full.record~.) + xlab("Number of daily cigarettes") + ylab("Density of responses") + ggtitle("TRUE = Full Records Available, FALSE = Missing Records")
dev.off()

# Linear model to see if you can predict whether the data have a full record based on cigar, omaps, fmaps
unclean.cig <- lm(full.record ~ cigar, ps1.data)
unclean.cig.om <- lm(full.record ~ omaps + cigar, ps1.data)
unclean.cig.om.fm <- lm(full.record ~ omaps + fmaps + cigar, ps1.data)

# The models seem to indicate you can predict whether there is a full record based on apgar and cigarette use....unfortunate.  
stargazer(unclean.cig.om.fm)

ps1.data.clean <- subset (ps1.data, full.record == TRUE)
ps1.data.missingvalues <- subset(ps1.data, full.record == FALSE)

print(nrow(ps1.data.clean)) #number of records remaining after cleaning

# Problem 1c: Summary table of clean data, write a new csv-------
var.labels <- attr(ps1.data, "var.labels")
var.labels[length(var.labels)+1] <- "full record present"

ps1.names <- data.frame("labels" = as.data.frame(var.labels))
colnames(ps1.names)[1] <- "labels" 

summarytable<-print(describe(ps1.data.clean, skew=FALSE, ranges=FALSE))

latex(title="variable", file="clean-summary.tex" , cbind(var.labels,summarytable), caption="Summary of clean Data", vbar=TRUE, size="footnotesize")

# stargazer(ps1.data.clean) # Doesn't work as well as the Hmisc version for this long table.

write.csv(ps1.data.clean, file = "ps1dataclean.csv")


#Problem 2a Simple difference in APGAR and birth weight -------

#'omaps' and 'fmaps' are the APGAR scores
#'dbrwt' is the birth weight in grams
# 'tobacco' is smoker status (1=yes, 2=no)

#change tobacco to factor and label values
ps1.data.clean$tobacco <- as.factor(ps1.data.clean$tobacco)
ps1.data.clean$tobacco <- revalue(ps1.data.clean$tobacco, c( "1" = "smoker", "2" = "nonsmoker" ))

smoke.impact <- ddply(ps1.data.clean, .(tobacco), summarize, 
                    mean.omaps = mean(omaps),
                    mean.fmaps = mean(fmaps),
                    mean.dbrwt = mean(dbrwt)
                    )

# conversion to character class for tobacco
smoke.impact$tobacco <- as.character(smoke.impact$tobacco)
# Add difference row
smoke.impact <- rbind(smoke.impact, c("difference", apply(smoke.impact[,2:4], 2, diff)))

stargazer(smoke.impact, summary=FALSE, digits = 2)

# # alt version 2a-------
# 
# smokers <- subset(ps1.data.clean, tobacco==1)
# nonsmokers <- subset(ps1.data.clean, tobacco==2)
# 
# smokerstats <- c(mean(smokers$omaps), mean(smokers$fmaps), mean(smokers$dbrwt))
# nonsmokerstats <- c(mean(nonsmokers$omaps), mean(nonsmokers$fmaps), mean(nonsmokers$dbrwt))
# meandif <- nonsmokerstats - smokerstats
# 
# smoketable <- matrix(c(smokerstats, nonsmokerstats, meandif), ncol=3, byrow=FALSE)
# colnames(smoketable) <- c("Mean Value (Infants with Smoker Mothers)", "Mean Value (Infants with Non-Smoker Mothers)", "Mean Difference between control and treatment")
# rownames(smoketable) <- c("one minute APGAR score", "five munute APGAR score", "birthweight")
# smoketable <- as.data.frame(smoketable)
# 
# 
# stargazer(smoketable, title = "Mean values of health figures in Infants with Smoker and Non-Smoker Mothers", type="latex")

# Problem 2b -------

#recode variables

ps1.data.clean$mrace3 <- as.factor(ps1.data.clean$mrace3)
ps1.data.clean$mrace3 <- revalue(ps1.data.clean$mrace3, c( "1" = "White", "2" = "Other", "3" = "Black" ))

ps1.data.clean$csex <- as.factor(ps1.data.clean$csex)
ps1.data.clean$csex <- revalue(ps1.data.clean$csex, c( "1" = "Male", "2" = "Female"))

ps1.data.clean$dplural <- as.factor(ps1.data.clean$dplural)
ps1.data.clean$dplural <- revalue(ps1.data.clean$dplural, c("1" = "Singleton", "2" = "Twin", "3"= "Triplet", "4" = "Quadruplet", "5" = "Quintuplet+"))

ps1.data.clean$alcohol <- as.factor(ps1.data.clean$alcohol)
ps1.data.clean$alcohol <- revalue(ps1.data.clean$alcohol, c("1" = "Drinker", "2" = "Nondrinker", "9" = "Unk."))

ps1.data.clean$dmar <- as.factor(ps1.data.clean$dmar)
ps1.data.clean$dmar <- revalue(ps1.data.clean$dmar, c("1" = "Married", "2" = "Unmarried"))

# T-tests for relationships.

print( t.test( omaps ~ tobacco, data = ps1.data.clean))
print( t.test( fmaps ~ tobacco, data = ps1.data.clean))
print( t.test( dbrwt ~ tobacco, data = ps1.data.clean))

#visual representation of relationship:

ps1.data.clean$all <- "all records" # flag for facets with all the same.

# birth weight - age
pdf(file="img/bw-age.pdf", width=5, height=7)
bw.age <- ggplot(ps1.data.clean, aes(dmage, dbrwt))
bw.age <- bw.age + 
  stat_smooth() +
  theme_bw() + 
  xlab("Maternal Age (years)") + 
  ylab("Birth Weight (grams)") + 
  facet_grid(all~.)

split.age <- ggplot(ps1.data.clean, aes(x=dmage))
split.age <- split.age +
  geom_density() + 
  theme_bw() + 
  xlab("Maternal Age (years)") +
  ylab("Density in Sub-sample") +
  facet_grid(tobacco~.)

arrange_ggplot2(bw.age, split.age, ncol=1)
dev.off()
  

# birth weight - marriage
pdf(file="img/bw-mar.pdf", width=5, height=7)

bw.mar <- ggplot(ps1.data.clean, aes(factor(dmar),dbrwt))
bw.mar <- bw.mar +
  geom_boxplot() +
  theme_bw() +
  xlab("Marital Status") +
  ylab("Birth Weight (grams)") +
  facet_grid(all~.)

split.mar <- ggplot(ps1.data.clean, aes(x=dmage))
split.mar <- split.age +
  geom_density() + 
  theme_bw() + 
  xlab("Maternal Age (years)") +
  ylab("Density in Sub-sample") +
  facet_grid(dmar~.)

arrange_ggplot2(bw.mar, split.mar, ncol=1)

dev.off()


# birth weight - maternal weight gain
pdf(file="img/bw-gain.pdf", width=5, height=7)
bw.gain <- ggplot(ps1.data.clean, aes(wgain, dbrwt))
bw.gain <- bw.gain + 
  stat_smooth() +
  theme_bw() + 
  xlab("Maternal Weight Gain (lbs)") + 
  ylab("Birth Weight (grams)") + 
  facet_grid(all~.)

split.gain <- ggplot(ps1.data.clean, aes(x=wgain))
split.gain <- split.gain +
  geom_density() + 
  theme_bw() + 
  xlab("Maternal Weight Gain (lbs)") +
  ylab("Density in Sub-sample") +
  facet_grid(tobacco~.)

arrange_ggplot2(bw.gain, split.gain, ncol=1)
dev.off()

## CrossTabs
ps1.xtab.data <- ps1.data.clean
# Add labels from ps1.data
for(i in 1:length(names(ps1.data))){
  label(ps1.xtab.data[[i]]) <- attr(ps1.data, "var.labels")[i]
}

# not-that-useful function for generating a list of crosstabs...
xtab.create <- function(data, const.col, cross.col, counts = FALSE){
# returns a data frame with percentiles of each cross column holding const column consant
# 
# data = a data frame
# const.col = column to be held constant (usually treatment)
# cross col = columns to vary (a concatenated list of character strings or indicies

# ERRORS
# TODO: Error handling
  # race of mother
xtab.out <- list()

for(factor in cross.col){
  xtab.factor <- CrossTable(data[[factor]],data[[const.col]])
  store.ver <- reshape(data=as.data.frame(xtab.factor$prop.row),idvar="x",direction="wide",timevar="y")
  if(counts){
    
    
  }
  xtab.out[[factor]] <- store.ver
}
return(xtab.out)
}



# # Improved labels for table (optional)
# label(ps1.data.clean$dmage) <- "Maternal Age (yr)"
# label(ps1.data.clean$tobacco) <- "Tobacco Use Status"
# label(ps1.data.clean$mrace3) <- "Maternal Race"
# label(ps1.data.clean$csex) <- "Infant Sex"
# label(ps1.data.clean$dplural) <- "Infant Plurality"
# label(ps1.data.clean$clingest) <- "Gestational Age (weeks)"
# label(ps1.data.clean$alcohol) <- "Alcohol Use Status"
# label(ps1.data.clean$phyper) <- "Preg. Hypertension"

# Grouped crosstabs using Hmisc

latex(summary( tobacco  ~ 
                 mrace3 + 
                 csex + 
                 dmar +
                 dplural + 
                 alcohol + 
                 phyper + 
                 chyper +
                 cardiac +
                 diabetes +
                 pre4000 +
                 dmeduc +
                 monpre +
                 dmage +
                 clingest + 
                 wgain, 
               data=ps1.xtab.data,  
               method="reverse", 
               overall=TRUE, long=TRUE
               ),
      title = "crosstab-tobacco",
      label = "tab:xtabTobacco",
      caption = "Contingency table for a range of factors by tobacco use status",
      exclude1=F
      )

# Problem 2c --------
# This one is all in latex doc.  Just describing things.

# Problem 2d ---------

sm.age <- lm(dbrwt ~ dmage, ps1.data.clean)

sm.age.mar <- lm(dbrwt ~ dmage + dmar, ps1.data.clean)

sm.a.m.w <- lm(dbrwt ~ dmage + dmar + wgain, ps1.data.clean)

sm.a.m.w.t <- lm(dbrwt ~ dmage + dmar + wgain + tobacco, ps1.data.clean)

sm.a.m.t <- lm(dbrwt ~ dmage + dmar + tobacco, ps1.data.clean)

sm.a.t <- lm(dbrwt ~ dmage + tobacco, ps1.data.clean)

sm.axm.t <- lm(dbrwt ~ dmage * dmar + tobacco, ps1.data.clean) #not used - look for cross of age:mar

# table without tobacco
stargazer(sm.age, sm.age.mar, sm.a.m.w, 
          type="latex",
          covariate.labels = c("Maternal Age", "Marital Status (unmarried)", "Weight Gain"),
          align = TRUE,
          style="qje", 
          single.row = FALSE,
          font.size="footnotesize",
          dep.var.labels = "Birth Weight",
          out = "combinedReg-noTob.tex"
          )

# table with tobacco
stargazer(sm.a.m.w.t, sm.a.m.t, sm.a.t, 
          type="latex",
          covariate.labels = c("Maternal Age", "Marital Status (unmarried)", "Weight Gain", "Tobacco (nonsmoker)"),
          align = TRUE,
          style="qje", 
          single.row = FALSE,
          font.size="footnotesize",
          dep.var.labels = "Birth Weight",
          out = "combinedReg-withTob.tex"
)




\end{verbatim}

\end{document}
