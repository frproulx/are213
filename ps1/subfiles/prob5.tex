We explored a dataset describing the birth outcomes for babies born in 1993 in Pennsylvania using the National Natality Data Files.  The effect we were exploring was the impact of maternal smoking on health outcomes, primarily with a focus on birth weight and APGAR scores, two metrics that are good indicators for overall natal health.  For the purposes of this work we take the ``treatment" to be smoking during the pregnancy only, so any covariate that occurs before the pregnancy is thought to be predetermined and not effected by the incidence of smoking.  Covariates that have to do with the pregnancy (such as visits to the Doctor) are confounded by the treatment.  

In PS1A we used the linear model to explore the data and found that APGAR scores were not sensitive to tobacco use in a significant way.  Birth weight, however, showed a trend.  The agnostic regression of differences in means without any accounting for covariates resulted in an ATE of 240 grams reduced birth weight for mothers who smoke.  There are, however, several predetermined covariates that we found were important to correct for, namely maternal age and marital status (others were considered but these two were the most important).  After correction for these factors (without interaction) the ATE is estimated at 200 grams.  

In PS1B we used non-parametric methods to further explore the data and create more robust estimates of the ATE for smoking.  Introducing a spline model for the maternal age - birth weight relationship did not result in a different ATE from the best one using parametric methods but adding interaction terms to the spline model did change the estimated ATE, from 200 to 220 grams.  Using a range of propensity score matching methods we confirmed that 220 grams is a reasonable estimate for the ATE and also estimate that the likelihood of very low birth weight is increased by 4 \% by smoking.  

Overall, the best estimate of ATE smoking is 220 grams, which was the result from a range of non-parametric approaches.  These approaches allow us to make a reasonable estimate of what the ATE would be if smoking were randomly assigned (which it is not).  The assumptions required for the non-parametric approaches (summarized above) are reasonable given the dataset.  The best estimate, however, is probably 200 grams.  This includes one significant figure and is confirmed by every statistical approach we took.  For the purposes of public health policymaking this estimate has a reasonable level of precision.