
\documentclass[letterpaper, 12pt]{article}

\usepackage{graphicx}
\usepackage{longtable}
\usepackage{rotating}
\usepackage{dcolumn}
\usepackage{listings}

% Code listing commands
\lstset{language=R,
basicstyle=\scriptsize\ttfamily,
commentstyle=\ttfamily,
numbers=left,
numberstyle=\footnotesize,
stepnumber=1,
numbersep=5pt,
showspaces=false,
showstringspaces=false,
showtabs=false,
frame=single,
tabsize=2,
captionpos=b,
breaklines=true,
breakatwhitespace=false,
title=\lstname,
escapeinside={},
keywordstyle={},
morekeywords={}
}

\begin{document}
\title{ARE213 Problem Set \#1B}
\author{Peter Alstone \& Frank Proulx}
\maketitle

\section{Problem \#1}
\subsection{Part A}
\emph{Under the assumption of random assignment conditional on the observables, what are the sources of misspecification bias in the estimates generated by the linear model estimated in Problem Set 1a?}

I think this is referring to the covariance between smoking and age.




\subsection{Part B}
\emph{Now, consider a series estimator. Estimate the smoking effects using a flexible functional form for the control variables (e.g., higher order terms and interactions). What are the benefits and drawbacks to this approach?}





We used R to complete this assignment.  The code is below:

\lstinputlisting{ps1b.R}

\lstinputlisting{../util/are213-func.R}


\end{document}
