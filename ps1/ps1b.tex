
\documentclass[letterpaper, 12pt]{article}

\usepackage{graphicx}
\usepackage{longtable}
\usepackage{rotating}
\usepackage{dcolumn}
\usepackage{listings}

% Code listing commands
\lstset{language=R,
basicstyle=\scriptsize\ttfamily,
commentstyle=\ttfamily,
numbers=left,
numberstyle=\footnotesize,
stepnumber=1,
numbersep=5pt,
showspaces=false,
showstringspaces=false,
showtabs=false,
frame=single,
tabsize=2,
captionpos=b,
breaklines=true,
breakatwhitespace=false,
title=\lstname,
escapeinside={},
keywordstyle={},
morekeywords={}
}

\begin{document}
\title{ARE213 Problem Set \#1B}
\author{Peter Alstone \& Frank Proulx}
\maketitle

\section{Problem \#1}
\subsection{Part A}
\emph{Under the assumption of random assignment conditional on the observables, what are the sources of misspecification bias in the estimates generated by the linear model estimated in Problem Set 1a?}

I think this is referring to the covariance between smoking and age.




\subsection{Part B}
\emph{Now, consider a series estimator. Estimate the smoking effects using a flexible functional form for the control variables (e.g., higher order terms and interactions). What are the benefits and drawbacks to this approach?}


\section{Problem \#2}
\subsection{Part A}

To calculate the propensity score, we estimated a logit model of mother's tobacco use (0=non-smoker, 1=smoker) as determined by the predetermined covariates shown in Table \ref{tab:propensities}.

\documentclass{article}
\usepackage{dcolumn}

\begin{document}

% Table created by stargazer v.4.5.1 by Marek Hlavac, Harvard University. E-mail: hlavac at fas.harvard.edu
% Date and time: Thu, Oct 17, 2013 - 07:47:06 AM
% Requires LaTeX packages: dcolumn 
\begin{table}[!htbp] \centering 
  \caption{Logistic function coefficients for propensity score models} 
  \label{tab:propensities} 
\begin{tabular}{@{\extracolsep{5pt}}lD{.}{.}{-3} D{.}{.}{-3} } 
\\[-1.8ex]\hline 
\hline \\[-1.8ex] 
\\[-1.8ex] & \multicolumn{2}{c}{Mother Tobacco-Use Status} \\ 
\\[-1.8ex] & \multicolumn{1}{c}{(1)} & \multicolumn{1}{c}{(2)}\\ 
\hline \\[-1.8ex] 
 Mother's Race not White or Black & -1.956^{***} & -1.954^{***} \\ 
  & (0.134) & (0.133) \\ 
  Mother's Years of Education & -0.817^{***} & -0.818^{***} \\ 
  & (0.028) & (0.028) \\ 
  Marital status & -0.205^{***} & -0.204^{***} \\ 
  & (0.005) & (0.005) \\ 
  Father's age & -1.256^{***} & -1.251^{***} \\ 
  & (0.022) & (0.021) \\ 
  Father's Years of Education & 0.029^{***} & 0.030^{***} \\ 
  & (0.002) & (0.001) \\ 
  Father Mexican & -0.131^{***} & -0.131^{***} \\ 
  & (0.005) & (0.005) \\ 
  Father Puerto Rican & -1.961^{***} & -1.957^{***} \\ 
  & (0.173) & (0.173) \\ 
  Father Cuban & -1.267^{***} & -1.268^{***} \\ 
  & (0.058) & (0.058) \\ 
  Father Central or South American & -0.567 & -0.567 \\ 
  & (0.364) & (0.364) \\ 
  Father Race Other or Unknown Hispanic & -1.933^{***} & -1.932^{***} \\ 
  & (0.205) & (0.205) \\ 
  Plurality of Infant & -0.890^{***} & -0.889^{***} \\ 
  & (0.120) & (0.120) \\ 
  Sex of Infant & -0.148^{***} &  \\ 
  & (0.054) &  \\ 
  Mother's age & -0.019 &  \\ 
  & (0.017) &  \\ 
  dmage & 0.003 &  \\ 
  & (0.002) &  \\ 
  Constant & 2.873^{***} & 2.707^{***} \\ 
  & (0.088) & (0.064) \\ 
 \textit{N} & \multicolumn{1}{c}{114,610} & \multicolumn{1}{c}{114,610} \\ 
Log Likelihood & \multicolumn{1}{c}{-44,310.690} & \multicolumn{1}{c}{-44,315.790} \\ 
Akaike Inf. Crit. & \multicolumn{1}{c}{88,651.370} & \multicolumn{1}{c}{88,655.580} \\ 
\hline 
\hline \\[-1.8ex] 
\textit{Notes:} & \multicolumn{2}{r}{$^{***}$Significant at the 1 percent level.} \\ 
 & \multicolumn{2}{r}{$^{**}$Significant at the 5 percent level.} \\ 
 & \multicolumn{2}{r}{$^{*}$Significant at the 10 percent level.} \\ 
\normalsize 
\end{tabular} 
\end{table} 
\end{document}



To test whether the propensity scores are comparable we perform a likelihood ratio test between the full and reduced model. 

NOTE from class: including insignificant terms can be beneficial in terms of getting better fit (by reducing 



\subsection{Part B}

This estimation assumes unconfoundedness. and conditional indpendence

\documentclass{article}
\usepackage{dcolumn}

\begin{document}

% Table created by stargazer v.4.5.1 by Marek Hlavac, Harvard University. E-mail: hlavac at fas.harvard.edu
% Date and time: Tue, Oct 15, 2013 - 07:18:58 AM
% Requires LaTeX packages: dcolumn 
\begin{table}[!htbp] \centering 
  \caption{Model of effects of tobacco use on birthweight using propensity score as a control} 
  \label{tab:propensitymodel} 
\footnotesize 
\begin{tabular}{@{\extracolsep{5pt}}lD{.}{.}{-3} } 
\\[-1.8ex]\hline 
\hline \\[-1.8ex] 
\\[-1.8ex] & \multicolumn{1}{c}{Mother Tobacco-Use Status} \\ 
\hline \\[-1.8ex] 
 Delta1 & -237.099^{***} \\ 
  & (9.180) \\ 
  & \\ 
 Beta & -241.076^{***} \\ 
  & (16.738) \\ 
  & \\ 
 Delta2 & 90.427^{***} \\ 
  & (34.496) \\ 
  & \\ 
 Constant & 3,445.897^{***} \\ 
  & (3.022) \\ 
  & \\ 
\textit{N} & \multicolumn{1}{c}{114,610} \\ 
R$^{2}$ & \multicolumn{1}{c}{0.025} \\ 
Adjusted R$^{2}$ & \multicolumn{1}{c}{0.025} \\ 
Residual Std. Error & \multicolumn{1}{c}{577.939 (df = 114606)} \\ 
F Statistic & \multicolumn{1}{c}{963.616$^{***}$ (df = 3; 114606)} \\ 
\hline 
\hline \\[-1.8ex] 
\textit{Notes:} & \multicolumn{1}{r}{$^{***}$Significant at the 1 percent level.} \\ 
 & \multicolumn{1}{r}{$^{**}$Significant at the 5 percent level.} \\ 
 & \multicolumn{1}{r}{$^{*}$Significant at the 10 percent level.} \\ 
\normalsize 
\end{tabular} 
\end{table} 
\end{document}



The estimated average treatment effect is given by 
\begin{align*}
\delta_1 + \delta_2 \overbar{p}(X_i)=
 -237.099 + (0.1594*90.427)=
-222.69
\end{align*}

This suggests that (pursuant to these assumptions) the average effect of smoking on birthweight is a reduction of 223 grams.

\subsection{Part C}





We used R to complete this assignment.  The code is below:

\lstinputlisting{ps1b.R}

\lstinputlisting{../util/are213-func.R}


\end{document}
