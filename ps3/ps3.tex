\documentclass[letterpaper, 12pt]{article}

\usepackage{graphicx}
\usepackage{longtable}
\usepackage{rotating}
\usepackage{dcolumn}
\usepackage{listings}
\usepackage{subfiles}
\usepackage{amsmath}

% Code listing commands
\lstset{language=R,
basicstyle=\scriptsize\ttfamily,
commentstyle=\ttfamily,
numbers=left,
numberstyle=\footnotesize,
stepnumber=1,
numbersep=5pt,
showspaces=false,
showstringspaces=false,
showtabs=false,
frame=single,
tabsize=2,
captionpos=b,
breaklines=true,
breakatwhitespace=false,
title=\lstname,
escapeinside={},
keywordstyle={},
morekeywords={}
}


\begin{document}
\title{ARE213 Problem Set \#3}
\author{Peter Alstone \& Frank Proulx}
\maketitle

\section{Problem 1: Linear models}

\subsection{Part A: LM results comparison}

\subfile{tab1a.tex}

Using a series of linear models (with heteroskedasticity consistent ``robust" standard errors), we find that for a range of model formulations there is a significant effect on housing price from the presence of hazardous waste cleanup sites, but not in the direction one would expect.  Instead of a reduction in housing value we estimate an increase in the value, which does not seem likely to be true.  The coefficient for the hazardous waste indicator variable (npl2000) takes a wide range of values depending on which additional explanatory variables are included in the model, from 0.04 (i.e., approximately a 4\% increase) for the simple model only including 1980 housing values and npl2000 to estimate 2000 housing values, to 0.09 for a model including both housing and demographic characteristics.  

\paragraph{Requirements for Unbiased Estimates [add to this]:}For our estimates to be unbiased we would need to include all of the potential sources of variation in housing price in a linear model.  A particular challenge is that there are very few sites with NPL2000 status (only 2\% of sites), so while the overall sample size is large there is very little support for estimates related to NPL2000 status compared to other covariates.  

\subsection{Part B: Comparing covariates}

\subfile{tab1b-1.tex}

\section{Problem 2: RDD setup}

\section{Problem 3: RDD First Stage}

\section{Problem 4: RDD Second Stage}

\section{Problem 5: Synthesis}

\section{Appendix: Code Listings}

\lstinputlisting{../util/are213-func.R}

\end{document}
